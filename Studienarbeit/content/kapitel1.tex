\chapter{Einführung}
\label{Kapitel1}

\section{Einleitung}

In einer Welt, die von immer rasanteren technologischen Fortschritten geprägt ist, hat die Robotik zweifellos eine zentrale Rolle eingenommen. Von Fertigungsanlagen bis hin zu autonomen Fahrzeugen – Roboter sind längst zu unverzichtbaren Akteuren in zahlreichen Branchen geworden. Dabei stellen sie nicht nur eine effiziente und präzise Alternative zur manuellen Arbeit dar, sondern bieten auch das Potenzial für beispiellose Innovationen. Eine dieser innovativen Anwendungen, die in den letzten Jahren verstärkt an Bedeutung gewonnen hat, ist die robotische Schleiftechnologie.

Roboterbasierte Schleifprozesse kommen in verschiedenen Bereichen der Industrie von entscheidender Bedeutung. Dazu zählen die Automobilindusrie, sowie die Luft- und Raumfahrt, aber auch die Medizintechnik. Diese Prozesse sind verschleißintensiv und erfordern eine kontinuierliche Überwachung und Wartung, um ihre Effizienz und Lebensdauer zu gewährleisten. In diesem Kontext gewinnt die akustische Vermessung als Methode zur prädiktiven Wartung von robotischen Schleifprozessen zunehmend an Bedeutung, da so die Prozesse automatisch und mit immer weniger menschlicher Arbeitskraft ausgeführt werden können.

Diese Studienarbeit widmet sich der Verbindung von Robotik und akustischer Messtechnik. Sie untersucht, wie akustische Signale, die während des Schleifvorgangs erzeugt werden, genutzt werden können, um den Zustand des Schleifwerkzeugs zu überwachen. Mit Hilfe dieser Daten kann der Schleifprozess kontrolliert und Anomalien frühzeitig erkannt werden, wodurch präventive Wartungsmaßnahmen eingeleitet werden können. Dies steigert nicht nur die die Produktivität, sondern erhöht auch die Sicherheit am Arbeitsplatz und senkt die nötigen Wartungskosten. Um dies zu erreichen erforscht die Arbeit die technischen Aspekte der akustischen Vermessung, beleuchtet aktuelle Fortschritte auf diesem Gebiet und untersucht, wie sie in die Praxis umgesetzt werden können. Wichtig ist hierbei zu erwähnen, dass die Arbeit kein ganzheitliches Konzept bietet, sondern sich vielmehr damit auseinandersetzt in einem spezifischen Anwendungsfall ein solche akustische Vermessung durchzuführen. Ziel ist es anhand dessen zukünftige Schritte abzuleiten, die die prädiktive Wartung weiter voranbringen können.

Die Wichtigkeit einer solchen Arbeit wird dadurch betont, dass bei einer solchen akustischen Vermessung zum einen viele Methoden genutzt werden können, zum anderen aber auch viele Probleme auftreten können. Solche Probleme reichen von der Positionierung des Mikrofon, da dieses sehr nah am Schleifer sein muss, bis hin zu komplexen Probleme in der Datenanalyse, bei welcher insbesondere durch die Anwendung von Fourier-Transformationen und Wavelet-Analysen verschiedenste Probleme auftreten können. 

\section{Schwerpunkt der Arbeit und Herangehensweise}

Der Schwerpunkt der Arbeit beschäftigt sich mit der Frage ob es möglich ist, nur anhand von Audioaufnahmen eines Schleifvorgangs die Details dessen automatisch herauszufinden. Dazu gehört vor allem die Drehgeschwindigkeit des Schleifers sowie der Anpressdruck auf das Werkstück. Die Hypothesen die dabei getestet werden sollen sind folgende:
\begin{enumerate}
    \item Es gibt eine Korrelation zwischen den im Audiosignal enthaltenen Frequenzen und der Drehgeschwindigkeit des Schleifers
    \item Es gibt eine Korrelation zwischen den im Audiosignal enthaltenen Frequenzen und dem Anpressdruck auf das Werkstück.
\end{enumerate}

Diese Hypothesen sollen getestet werden, indem Audiodaten analysiert werden, die unter verschiedenen Schleifbedingungen gesammelt werden. Die Kriterien für die Auswahl der Schleifbedingungen umfassen z. B. unterschiedliche Drehgeschwindigkeiten und gewährleisten so eine breitere Datenbasis. Die Audiodaten wurden mittels eines hoch sensiblen Mikrofons und der Software Audacity aufgenommen. Eine speziell dafür konstruierte Mikrofonhalterung wurde verwendet, um einerseits konstante Aufnahmebedingungen zu gewährleisten und andererseits ein Spielraum für die Entfernung des Mikrofons zum Schleifkopf zu geben. Eine genauere Erläuterung dieses Versuchsaufbaus und der Datenaufnahme findet im Kapitel \ref{Kapitel5} statt. Anschließend ist es wichtig die Rohdaten aufzubereiten, um Rauschen und unerwünschte Störungen zu eliminieren, Grundlagen hierfür werden im Kapitel \ref{Kapitel2} vermittelt. Für die Datenanalyse werden verschiedene Methoden wie die Fourier- und Wavelet-Transformation in Betracht bezogen. Die Grundlagen dieser Methoden wird zunächst in Kapitel \ref{Kapitel2} erläutert. Das Kapitel \ref{Kapitel3} baut auf diesen Grundlagen auf und erläutert wie die Methoden sich im Laufe der Zeit entwickelt haben und was der aktuelle Stand dieser ist. Diese State-of-the-Art Methoden liefern die Grundlagen für die Durchführung der tatsächlichen Analyse, welche im Kapitel \ref{Kapitel6} genauer erläutert wird. Da es sich bei dem untersuchten Problem um einen robotischen Prozess handelt, wird anschließend an die Analyse eine ROS-Node implementiert, welche die Methoden zur Analyse benutzt und eine Auswertung in ROS ermöglicht (Kapitel \ref{Kapitel7}). Der Abschluss der Arbeit erfolgt in den Kapiteln \ref{Kapitel8} und \ref{Kapitel9}. Zunächst wird ein Fazit gezogen, das die erzielten Ergebnisse sowie aufgetretene Probleme zusammenfasst. Anschließend wird ein Ausblick auf die Zukunft gegeben. Auf Grundlage der gewonnenen Erkenntnisse werden Ideen und Vorschläge präsentiert, um die akustische Vermessung eines robotischen Schleifprozesses weiter zu verbessern. 


%%%%%%%%%%%%%%%%%%%%%%%%%%%%%%%%%%%%%%%%%%%%%%%%%%%%%%%%%%%%%%%%%%%%%%%%%%%%%%%
\endinput
%%%%%%%%%%%%%%%%%%%%%%%%%%%%%%%%%%%%%%%%%%%%%%%%%%%%%%%%%%%%%%%%%%%%%%%%%%%%%%%
