\chapter{Ausblick}
\label{Kapitel9}

Die vorliegenden Ergebnisse und deren kritische Betrachtung legen nahe, dass es mehrere vielversprechende Forschungsrichtungen gibt, um die prädiktive Wartung weiter voranzutreiben. Diese Themen könnten in zukünftigen Arbeiten vertieft werden:

\paragraph{Erweiterung der Datengrundlage}

Um die Robustheit und Generalisierbarkeit der Analyseverfahren zu verbessern, könnten zukünftige Studien eine umfangreichere Datengrundlage schaffen. Dies beinhaltet:
\begin{itemize}
    \item \textbf{Unterschiedliche Materialien}: Die Aufnahme und Analyse von Audiosignalen bei verschiedenen Materialtypen würde helfen, die Algorithmen auf eine breitere Anwendungsbasis zu erweitern.
    \item \textbf{Verschiedene Umgebungsbedingungen}: Datenaufnahmen unter variierenden Umweltbedingungen wie Temperatur, Luftfeuchtigkeit und Umgebungsgeräuschen könnten die Robustheit der Analyse verbessern.
    \item \textbf{Langzeitstudien}: Langfristige Beobachtungen und Datenaufnahmen könnten helfen, die Stabilität und Zuverlässigkeit der Analyse über längere Zeiträume zu gewährleisten.
\end{itemize}

\paragraph{Verbesserung der Analysemethoden}

Die verwendeten Audioanalysemethoden könnten durch fortschrittlichere Techniken ergänzt oder ersetzt werden:
\begin{itemize}
    \item \textbf{Künstliche Intelligenz und Maschinelles Lernen}: Der Einsatz von KI und maschinellem Lernen, insbesondere Deep Learning, könnte die Erkennung und Analyse komplexer Muster in den Audiodaten verbessern. Eine größere und vielfältigere Datengrundlage würde es ermöglichen, Modelle zu trainieren, die verlässliche Vorhersagen treffen können.
    \item \textbf{Hybride Modelle}: Eine Kombination aus traditionellen Signalverarbeitungstechniken und modernen KI-Ansätzen könnte die Stärken beider Methoden vereinen und zu präziseren Ergebnissen führen.
    \item \textbf{Automatisierte Labeling-Methoden}: Weiterentwicklungen in der automatisierten Datenannotation könnten den Bedarf an manuellem Labeln verringern und die Effizienz der Datenauswertung steigern. Hierfür können die bereits erzielten Ergebnisse genutzt und ausgebaut werden.
\end{itemize}

Zusammenfassend bietet die Weiterentwicklung der prädiktiven Wartung durch erweiterte Datenanalyse, den Einsatz moderner KI-Technologien und die Integration in industrielle Prozesse ein großes Potenzial zur Steigerung der Effizienz und Zuverlässigkeit in der Fertigung. Zukünftige Arbeiten sollten diese Bereiche weiter erforschen und die prädiktive Wartung somit auf ein neues Level heben.


%%%%%%%%%%%%%%%%%%%%%%%%%%%%%%%%%%%%%%%%%%%%%%%%%%%%%%%%%%%%%%%%%%%%%%%%%%%%%%%
\endinput
%%%%%%%%%%%%%%%%%%%%%%%%%%%%%%%%%%%%%%%%%%%%%%%%%%%%%%%%%%%%%%%%%%%%%%%%%%%%%%%