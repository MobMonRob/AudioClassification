\begin{abstract}

Diese Arbeit untersucht die Schnittstelle zwischen Robotik und akustischer Messtechnik und konzentriert sich dabei insbesondere auf die Nutzung akustischer Signale, die während Schleifprozessen erzeugt werden, um den Zustand von Schleifwerkzeugen zu überwachen. Durch die Nutzung dieser akustischen Signale können der Schleifprozess gesteuert und Anomalien frühzeitig erkannt werden, was präventive Wartungsmaßnahmen ermöglicht. Dieser Ansatz steigert nicht nur die Produktivität, sondern verbessert auch die Sicherheit am Arbeitsplatz und senkt die Wartungskosten.

Die Studie untersucht, ob es möglich ist, Details des Schleifprozesses, wie etwa die Drehzahl der Schleifmaschine und den ausgeübten Druck auf das Werkstück, allein aus Audioaufzeichnungen automatisiert zu ermitteln. Im Rahmen einer experimentellen Studie wurden Audiodaten unter verschiedenen Schleifbedingungen gesammelt und mit Methoden wie Fourier- und Wavelet-Transformationen analysiert. Die Ergebnisse deuten auf Korrelationen zwischen den im Audiosignal vorhandenen Frequenzen und sowohl der Rotationsgeschwindigkeit als auch dem ausgeübten Druck auf das Werkstück hin.

Darüber hinaus werden in der Arbeit die Herausforderungen bei der Positionierung des Mikrofons in der Nähe des Schleifers und die Komplexität der Datenanalyse erörtert, einschließlich der Probleme, die bei Fourier- und Wavelet-Transformationen auftreten. Die Studie bietet zwar keine umfassende Lösung, bietet aber Einblicke in zukünftige Schritte zur Weiterentwicklung der vorausschauenden Wartung bei Roboterschleifanwendungen. Die Forschung unterstreicht das Potenzial für die Weiterentwicklung von Methoden zur Extraktion relevanter Merkmale aus Audiosignalen zur Unterstützung prädiktiver Wartungsstrategien.

\end{abstract}