%%%%%%%%%%%%%%%%%%%%%%%%%%%%%%%%%%%%%%%%%%%%%%%%%%%%%%%%%%%%%%%%%%%%%%%%%%%%%%%
%% Descr:       LaTeX Style für Vorlage für Berichte der DHBW-Karlsruhe
%% Author:      Prof. Dr. Jürgen Vollmer, juergen.vollmer@dhbw-karlsruhe.de
%% $Id: bericht.sty,v 1.25 2020/03/13 15:38:13 vollmer Exp $
%% -*- coding: utf-8 -*-
%%%%%%%%%%%%%%%%%%%%%%%%%%%%%%%%%%%%%%%%%%%%%%%%%%%%%%%%%%%%%%%%%%%%%%%%%%%%%%%

%% ACHTUNG, wenn man eine eigene Format-Datei benutzt dann werden Änderungen an bericht.sty
%% erst wirksam, wenn die Format-Datei neu erzeugt wurde!!!

%\NeedsTeXFormat{LaTeX2e}
%\ProvidesPackage{bericht}[2017/10/06 v1.2]

% Bitte die Codierung Ihrer Dateien auswählen:
% \usepackage[latin1]{inputenc}    % Für UNIX mit ISO-LATIN-codierten Dateien
% \usepackage[applemac]{inputenc}  % Für Apple Mac
% \usepackage[ansinew]{inputenc}   % Für Microsoft Windows
\usepackage[utf8]{inputenc}        % UTF-8 codierte Dateien
                                   % Dieses Dokument ist unter Unix erstellt, daher
                                   % wird diese Input-Codierung benutzt.

%%%%%%%%%%%%%%%%%%%%%%%%%%%%%%%%%%%%%%%%%%%%%%%%%%%%%%%%%%%%%%%%%%%%%%%%%%%
%% needed packages
%%%%%%%%%%%%%%%%%%%%%%%%%%%%%%%%%%%%%%%%%%%%%%%%%%%%%%%%%%%%%%%%%%%%%%%%%%%
\usepackage[onehalfspacing]{setspace}
\usepackage[utf8]{inputenc}   % UTF8-Codierung
\usepackage{babel}      % Sprachanpassungen für generierte Texte wie "Inhaltsverzeichnis" etc
\usepackage[T1]{fontenc}% Interne LaTeX Codierungen
\usepackage[dvipsnames,table]{xcolor} % Extending L A TEX’s color facilities
\usepackage[babel, german=guillemets]{csquotes}   % Context sensitive quotation facilities.
\usepackage{xspace}     % http://www.ctan.org/tex-archive/help/Catalogue/entries/xspace.html
\usepackage{array}      % http://www.ctan.org/tex-archive/help/Catalogue/entries/array.html
\usepackage{tabularx}   % http://www.ctan.org/tex-archive/help/Catalogue/entries/tabularx.html
\usepackage{eurosym}    % \euro
\usepackage{pdfpages}   % http://www.ctan.org/tex-archive/help/Catalogue/entries/pdfpages.html
\usepackage{needspace}  % http://www.tex.ac.uk/cgi-bin/texfaq2html?label=nopagebrk
\usepackage[bookmarksopen,bookmarksnumbered]{hyperref}
\usepackage{bookmark}   % Bookmarks for hyperref
\usepackage{graphicx}
\usepackage[headings]{fullpage}
\usepackage{fancyhdr}
\usepackage{url}
\usepackage{microtype}  % http://tug.ctan.org/tex-archive/macros/latex/contrib/microtype/
\usepackage{lmodern}    % Computern-Modern Schriftfamilie
\usepackage{amssymb}    % Symbole
\usepackage{framed}     % Framed or shaded regions that can break across pages.
		        % http://dante.ctan.org/tex-archive/help/Catalogue/entries/framed.html
			% Benutzung siehe erklaerung.tex

\usepackage{wrapfig}    % Bilder von textumfliessen lassen

\usepackage[colorinlistoftodos]{todonotes}
                        % Einfache Verwaltung und Erstellung von TODO's Markierungen
			% http://tug.ctan.org/tex-archive/macros/latex/contrib/todonotes/
			% wichtige Paket-Optionen: disable

\usepackage{makeidx}    % Erstellung eines Indexes
\makeindex

%%%%%%%%%%%%%%%%%%%%%%%%%%%%%%%%%%%%%%%%%%%%%%%%%%%%%%%%%%%%%%%%%%%%%%%%%%%
%% eigene Macros
%%%%%%%%%%%%%%%%%%%%%%%%%%%%%%%%%%%%%%%%%%%%%%%%%%%%%%%%%%%%%%%%%%%%%%%%%%%

\newcommand{\email}[1]{\href{mailto:#1}{\textless#1\textgreater}}

%%%%%%%%%%%%%%%%%%%%%%%%%%%%%%%%%%%%%%%%%%%%%%%%%%%%%%%%%%%%%%%%%%%%%%%%%%%
%% citation, bibliography, BIBLATEX
%%%%%%%%%%%%%%%%%%%%%%%%%%%%%%%%%%%%%%%%%%%%%%%%%%%%%%%%%%%%%%%%%%%%%%%%%%%
% Wer etwas mehr "Kontrolle" über das Aussehen der Referenzen haben möchte, kann
% auch das "biblatex"-Paket benutzen.
\usepackage{etoolbox}  % http://dante.ctan.org/tex-archive/help/Catalogue/entries/etoolbox.html
\usepackage[
	hyperref=true,          % Klickbare Referenzen in der PDF-Datei
        backref=true,           % In der Literaturref. die Seiten angeben, wo ein \cite dazu steht
        bibencoding=inputenc,   % s. inputenc-Paket
	style=authoryear-comp,
        backend=bibtex,
	sorting=nty]{biblatex}  % http://dante.ctan.org/tex-archive/help/Catalogue/entries/biblatex.html
%\renewcommand{\mkbibnamelast}[1]{\textsc{#1}}

% Abstand der einzelnen Einträge in der Literaturangaben
\renewcommand{\bibitemsep}{1ex}

% Welche Klammern soll \parencite{..} benutzen?
\renewcommand{\bibleftparen}{[}
\renewcommand{\bibrightparen}{]}

% biblatex's \cite{..} gibt "normalerweise" keine Klammern um die Referenz aus
% \parencite{..} gibt Klammern aus (die oben definiert sind)
% Damit das "normale" Verhalten andere BibTeX-Stile realisiert wird, d.h. \cite{..}
% gibt Klammen aus, wird folgendes definiert:
% "Merke" ursprünliche Definition unter neuem Namen
\let\citeNoParen\cite
% "Redefiniere" \cite:
\let\cite\parencite

% Meine speziellen \cite-Kommandos:
% Ausgabe des Untertitels (SUBTITLE) Feldes einer Referenz
\DeclareCiteCommand{\citesubtitle}
  {\boolfalse{citetracker}%
   \boolfalse{pagetracker}%
   \usebibmacro{prenote}}
  {\indexfield{indextitle}%
   \printfield[citetitle]{subtitle}}
  {\multicitedelim}
  {\usebibmacro{postnote}}

% Ausgabe des Titels (TITLE) Feldes und der Referenz
\DeclareCiteCommand{\citetitleref}
  {\booltrue{citetracker}%
   \booltrue{pagetracker}%
   \usebibmacro{prenote}}
  {\indexfield{indextitle}%
   \printfield[citetitle]{title} \cite{\thefield{entrykey}}}
  {\multicitedelim}
  {\usebibmacro{postnote}}

% wie citetitleref, nur als Fussnote
\DeclareCiteCommand{\citetitlerefFootnote}
  {}
  {\footnote{\citetitleref{\thefield{entrykey}}}}
  {}
  {}


% Ausgabe der URL im HOWPUBLISHED Feld
% Referenzieren von URL's, Format in der *.bib-Datei
%@MISC{key,
%  AUTHOR	= "....",
%  TITLE	= "Webseite....",
%  HOWPUBLISHED = "\url{http://www.domain.tld}",
%  YEAR		= YYYY,    % Jahr  der Einsichtname, YYYY = Jahreszahl 4 Stellog
%  MONTH	= ABC      % Monat der Einsichtnahme (jan, feb, mar, apr, may, jun, jul, aug, sep, oct, nov. dec
%}
\DeclareCiteCommand{\citeurl}
  {\booltrue{citetracker}%
   \booltrue{pagetracker}%
   \usebibmacro{prenote}}
  {\indexfield{indextitle}%
   \printfield[citetitle]{howpublished}}
  {\multicitedelim}
  {\usebibmacro{postnote}}

% Ausgabe der URL im HOWPUBLISHED Feld und Referenz
\DeclareCiteCommand{\citeurlref}
  {\booltrue{citetracker}%
   \booltrue{pagetracker}%
   \usebibmacro{prenote}}
  {\indexfield{indextitle}%
   \printfield[citetitle]{howpublished} \cite{\thefield{entrykey}}}
  {\multicitedelim}
  {\usebibmacro{postnote}}

% Augabe: Vorname Nachname des Autors
\DeclareCiteCommand{\citefullauthor}
  {\booltrue{citetracker}%
   \booltrue{pagetracker}%
   \usebibmacro{prenote}}
  {\indexfield{indextitle}%
    \textsc{\printnames[byeditor]{author}}}
  {\multicitedelim}
  {\usebibmacro{postnote}}

%%%%%%%%%%%%%%%%%%%%%%%%%%%%%%%%%%%%%%%%%%%%%%%%%%%%%%%%%%%%%%%%%%%%%%%%%%%
% Programmlistings setzen
%%%%%%%%%%%%%%%%%%%%%%%%%%%%%%%%%%%%%%%%%%%%%%%%%%%%%%%%%%%%%%%%%%%%%%%%%%%
\usepackage{listings}   % http://www.ctan.org/tex-archive/macros/latex/contrib/listings/

% Wie sollen die Überschriften benannt werden:
\renewcommand{\lstlistingname}{Algorithmus}

% Wie die Liste der Listings, s. \lstlistoflistings in bericht.tex
\renewcommand{\lstlistlistingname}{Liste der Algorithmen}

% So kann man einen Stil für alle  Algorithmen definieren
\lstdefinestyle{algoBericht}{
  numbers=left,              % Zeilennummern einfügen
  numberstyle=\tiny,         % wie werden sie gesetzt
  numbersep=5pt,             % Abstand der Nummern zum Text
  numberblanklines=false,    % bei Leerzeilen keine Nummer ausgeben (aber zählen)
  basicstyle=\sffamily\small,         % Wie soll der Algorithmus gesetzt werden
}


%%%%%%%%%%%%%%%%%%%%%%%%%%%%%%%%%%%%%%%%%%%%%%%%%%%%%%%%%%%%%%%%%%%%%%%%%%%
% Abkürzungen, http://www.ctan.org/tex-archive/macros/latex/contrib/acronym/
%%%%%%%%%%%%%%%%%%%%%%%%%%%%%%%%%%%%%%%%%%%%%%%%%%%%%%%%%%%%%%%%%%%%%%%%%%%
\usepackage[printonlyused,withpage]{acronym}

%%%%%%%%%%%%%%%%%%%%%%%%%%%%%%%%%%%%%%%%%%%%%%%%%%%%%%%%%%%%%%%%%%%%%%%%%%%
% Formelverzeichnis, Dank an Andy Nöltner <ANoeltner@lstelcom.com>
% Leider versursacht float zusammen mit Hyperref Warnungen
% siehe http://www.tex.ac.uk/cgi-bin/texfaq2html?label=hyperdupdest
% Wenn also das Formelverzeichnis nicht benötigt, dann das folgende
% auskommentieren
%%%%%%%%%%%%%%%%%%%%%%%%%%%%%%%%%%%%%%%%%%%%%%%%%%%%%%%%%%%%%%%%%%%%%%%%%%%
\usepackage{float}
\newfloat{eq}{H}{for}[chapter]
\newcommand{\forname}{Formelverzeichnis}
\newcommand{\listofequations}{\listof{eq}{\forname}}

\newcommand{\eqlabel}[2]{
        \label{#1}
        \addcontentsline{for}{eq}{(\ref{#1}) #2}}

%%%%%%%%%%%%%%%%%%%%%%%%%%%%%%%%%%%%%%%%%%%%%%%%%%%%%%%%%%%%%%%%%%%%%%%%%%%

% zum zeichnen
\usepackage{tikz}
\usetikzlibrary{shapes.geometric}
\usetikzlibrary{arrows.meta}
\usetikzlibrary{positioning}
\usetikzlibrary{calc}

\usepackage{pgfplots}
\pgfplotsset{width=10cm,compat=1.9}

% mathe
\usepackage{amsmath}

%%%%%%%%%%%%%%%%%%%%%%%%%%%%%%%%%%%%%%%%%%%%%%%%%%%%%%%%%%%%%%%%%%%%%%%%%%%

% Definitionen mit Ausgabe im Index
% gibt #2 aus und fügt #1 bzw. #2 (wenn #1 nicht angegeben) in den Index ein
\newcommand{\Def}[2][]{%
   \def\OPTARG{#1}%
   \def\EMPTY{}%
   \ifx\OPTARG\EMPTY\index{#2}\else\index{#1}\fi%
   \textbf{#2}\xspace%
}

%%%%%%%%%%%%%%%%%%%%%%%%%%%%%%%%%%%%%%%%%%%%%%%%%%%%%%%%%%%%%%%%%%%%%%%%%%%
\endinput
%%
%% End of file `bericht.sty'.
